\documentclass[conference]{IEEEtran}
\IEEEoverridecommandlockouts

\usepackage{cite}
\usepackage{amsmath,amssymb,amsfonts}
\usepackage{graphicx}
\usepackage{textcomp}
\usepackage{xcolor}
\def\BibTeX{{\rm B\kern-.05em{\sc i\kern-.025em b}\kern-.08em
    T\kern-.1667em\lower.7ex\hbox{E}\kern-.125emX}}

\begin{document}

\title{Mixed-Integer Convex Programming for \\Motion Planning in Dual-Arm Manipulation}

\author{\IEEEauthorblockN{Phone Thiha Kyaw and Karyna Volokhatiuk}
\IEEEauthorblockA{\textit{University of Toronto Institute for Aerospace Studies} \\
Email: \{phone.thiha, karyna.volokhatiuk\}@robotics.utias.utoronto.ca}
}

\maketitle

\begin{abstract}
Motion planning for high-degree-of-freedom robotic manipulators is challenging due to complex constraints such as collision avoidance, kinematic limits, and trajectory smoothness.
%
In this project, we propose formulating the motion planning problem for dual-arm manipulation (14-DOF) as a convex optimization problem.
%
Specifically, we will leverage the recently studied Graph of Convex Sets (GCS) framework~\cite{marcucci2024shortest} to model motion planning as a compact mixed-integer optimization problem, similar to~\cite{marcucci2023motion}.
%
To evaluate its effectiveness, we will compare our approach against classical sampling-based motion planners from OMPL~\cite{sucan2012open}.
%
The proposed method will be implemented and tested in Drake~\cite{drake}, a widely used toolbox for modeling and simulating robotic systems.
\end{abstract}

\section{Introduction}

\section{Related Work}

\section{Proposed Approach}

\bibliographystyle{IEEEtran}
\bibliography{IEEEabrv,main}

\end{document}
